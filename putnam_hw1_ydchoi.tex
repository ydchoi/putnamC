\documentclass{article} % For LaTeX2e
\usepackage{nips14submit_e,times}
\usepackage{amsmath}
\usepackage{amsthm}
\usepackage{amssymb}
\usepackage{mathtools}
\usepackage{hyperref}
\usepackage{url}
\usepackage{algorithm}
\usepackage[noend]{algpseudocode}
%\documentstyle[nips14submit_09,times,art10]{article} % For LaTeX 2.09

\usepackage{graphicx}
\usepackage{caption}
\usepackage{subcaption}

\def\eQb#1\eQe{\begin{eqnarray*}#1\end{eqnarray*}}

\providecommand{\e}[1]{\ensuremath{\times 10^{#1}}}
\providecommand{\pb}[0]{\pagebreak}


\newenvironment{claim}[1]{\par\noindent\underline{Claim:}\space#1}{}
\newtheoremstyle{quest}{\topsep}{\topsep}{}{}{\bfseries}{}{ }{\thmname{#1}\thmnote{ #3}.}
\theoremstyle{quest}
\newtheorem*{definition}{Definition}
\newtheorem*{theorem}{Theorem}
\newtheorem*{question}{Question}
\newtheorem*{exercise}{Exercise}
\newtheorem*{challengeproblem}{Challenge Problem}
\newtheorem*{solution}{Solution}
\usepackage{verbatimbox}
\usepackage{listings}
\title{Putnam: Assignment I}


\author{
Youngduck Choi \\
Courant Institute of Mathematical Sciences\\
New York University \\
\texttt{yc1104@nyu.edu} \\
}


% The \author macro works with any number of authors. There are two commands
% used to separate the names and addresses of multiple authors: \And and \AND.
%
% Using \And between authors leaves it to \LaTeX{} to determine where to break
% the lines. Using \AND forces a linebreak at that point. So, if \LaTeX{}
% puts 3 of 4 authors names on the first line, and the last on the second
% line, try using \AND instead of \And before the third author name.

\newcommand{\fix}{\marginpar{FIX}}
\newcommand{\new}{\marginpar{NEW}}

\nipsfinalcopy % Uncomment for camera-ready version

\begin{document}


\maketitle

\begin{abstract}
Analysis problems
\end{abstract}

\section{Solutions to the problems}

\begin{question}[1.1. Convergence of an Improper Integral]
\end{question}
\begin{solution}
We wish to show that the given improper integral $\underset{B\to\infty}{\lim}\int_{0}^{B} sin(x)sin(x^2)dx$
is convergent. Observe that $\int_{0}^{1} sin(x)sin(x^2) dx$ achieves a finite value,
as the absolute values of the integrand over the interval $[0,1]$ 
is bounded by $1$. Hence,
it suffices to show that $\int_{1}^{B} sin(x)sin(x^2)dx$ is convergent. 
We integrate $\int_{1}^{B} sin(x)sin(x^2)dx$ by parts. 
To this end, we set $u=\dfrac{sin(x)}{2x}$ and 
$dv = sin(x^2)2x dx$, so that $du = \dfrac{1}{2}(\dfrac{cos(x)}{x} - \dfrac{sin(x)}{x^2}) dx$ 
and $v = -cos(x^2)$. It follows that
\eQb
\int_{1}^{B} sin(x)sin(x^{2}) dx &=& -\dfrac{sin(x)}{2x}cos(x^{2})|_{1}^{B} + \dfrac{1}{2}
\int_{1}^{B} cos(x^{2})(\dfrac{cos(x)}{x} - \dfrac{sin(x)}{x^{2}})dx. \\
\eQe
Simplifying the second integral, we obtain
\eQb
\int_{1}^{B} sin(x)sin(x^{2}) dx &=& -\dfrac{sin(x)}{2x}cos(x^{2})|_{1}^{B} 
- \dfrac{1}{2}\int_{1}^{B}\dfrac{sin(x)cos(x^2)}{x^{2}})dx \\
&+& \dfrac{1}{2} \int_{1}^{B} \dfrac{cos(x)cos(x^2)}{x}dx. 
\eQe
As $B \to \infty$ we see that $-\dfrac{sin(x)}{2x}cos(x^2)|_{1}^{B}$ tends to $0$ and the 
$\int_{1}^{B} \dfrac{sin(x)cos(x^2)}{x^2} dx$ can be shown to be absolutely convergent, thus
convergent via the comparison test for improper integrals with $\dfrac{1}{2}\int_{1}^{B} 
\dfrac{1}{x^2}dx$, it only remains to show that $\int_{1}^{B} \dfrac{cos(x)(cos(x^2)}{x}dx$
term is convergent. We now integrate $\int_{1}^{B} \dfrac{cos(x)cos(x^2)}{x}dx$ by parts. 
We set $u=\dfrac{cos(x)}{2x^2}$ and $dv = cos(x^2)2x dx$, so that 

\pb

$du = -\dfrac{1}{2}(\dfrac{sin(x)}{x^2} + \dfrac{2cos(x)}{x^3})dx $ 
and $v = sin(x^2)$. It follows that
\eQb
\int_{1}^{B} \dfrac{cos(x)cos(x^2)}{x}dx = \dfrac{cos(x)}{2x^2}sin(x^2)|_{1}^{B} + 
\dfrac{1}{2}\int_{1}^{B} sin(x^2)(\dfrac{sin(x)}{x^2} + \dfrac{2cos(x)}{x^3})dx. 
\eQe
Simplifying the second integral, we have
\eQb
\int_{1}^{B} \dfrac{cos(x)cos(x^2)}{x}dx &=& \dfrac{cos(x)}{2x^2}sin(x^2)|_{1}^{B} + 
\dfrac{1}{2}\int_{1}^{B} \dfrac{sin(x^2)sin(x)}{x^2} \\
&+& \int_{1}^{B}\dfrac{sin(x^2)cos(x)}{x^3}dx. 
\eQe
As $B \to \infty$, we see that $\dfrac{cos(x)}{2x^2}sin(x^2)|_1^B$ tends to $0$, and 
both integrals can be shown to be absolutely convergent, thus convergent via the 
comparison test with $\int_{1}^{B}\dfrac{1}{x^2}dx$ and 
$\int_{1}^{B}\dfrac{1}{x^3}dx$ respectively. Therefore, we have proven that
the improper integral $\underset{B \to \infty}{\lim}\int_{0}^{B} sin(x)sin(x^2)dx$ is
convergent. $\qed$ 
\end{solution}
\bigskip

\begin{question}
\end{question}
\begin{solution}
We want to show that, for all integers $n > 1$, the following inequality holds:
\eQb
\dfrac{1}{2ne} < \dfrac{1}{e} - (1 - \dfrac{1}{n})^n < \dfrac{1}{ne}.
\eQe
Multiplying by $e$ and subtracting $1$ from all sides of the inequality we can 
rewrite the inequality as 
\eQb
\dfrac{1}{2n} - 1 < -e(1 - \dfrac{1}{n})^n < \dfrac{1}{n} - 1.
\eQe
Multiplying by $-1$ and taking the log to all sides, we can again rewrite the 
inequality as 
\eQb
log(1 - \dfrac{1}{2n}) > -1 + nlog(1 - \dfrac{1}{n}) > log(1 - \dfrac{1}{n}). 
\eQe
Now, we can rewrite the above inequality with the terms of 
the Taylor expansion of $log(1 - x)$ as
\eQb
-\sum_{i=1}^{\infty} \dfrac{1}{i2^i n^i} > 
-\sum_{i=1}^{\infty} \dfrac{1}{(i+1)n^i} >
-\sum_{i=1}^{\infty} \dfrac{1}{in^i},
\eQe
which can be seen to hold as the inequality holds term by term. $\qed$
  
\end{solution}

\end{document}
