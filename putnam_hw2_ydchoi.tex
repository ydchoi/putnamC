\documentclass{article} % For LaTeX2e
\usepackage{nips14submit_e,times}
\usepackage{amsmath}
\usepackage{amsthm}
\usepackage{amssymb}
\usepackage{mathtools}
\usepackage{hyperref}
\usepackage{url}
\usepackage{algorithm}
\usepackage[noend]{algpseudocode}
%\documentstyle[nips14submit_09,times,art10]{article} % For LaTeX 2.09

\usepackage{graphicx}
\usepackage{caption}
\usepackage{subcaption}

\def\eQb#1\eQe{\begin{eqnarray*}#1\end{eqnarray*}}

\providecommand{\e}[1]{\ensuremath{\times 10^{#1}}}
\providecommand{\pb}[0]{\pagebreak}


\newenvironment{claim}[1]{\par\noindent\underline{Claim:}\space#1}{}
\newtheoremstyle{quest}{\topsep}{\topsep}{}{}{\bfseries}{}{ }{\thmname{#1}\thmnote{ #3}.}
\theoremstyle{quest}
\newtheorem*{definition}{Definition}
\newtheorem*{theorem}{Theorem}
\newtheorem*{question}{Question}
\newtheorem*{exercise}{Exercise}
\newtheorem*{challengeproblem}{Challenge Problem}
\newtheorem*{solution}{Solution}
\usepackage{verbatimbox}
\usepackage{listings}
\title{Putnam: Assignment II}


\author{
Youngduck Choi \\
Courant Institute of Mathematical Sciences\\
New York University \\
\texttt{yc1104@nyu.edu} \\
}


% The \author macro works with any number of authors. There are two commands
% used to separate the names and addresses of multiple authors: \And and \AND.
%
% Using \And between authors leaves it to \LaTeX{} to determine where to break
% the lines. Using \AND forces a linebreak at that point. So, if \LaTeX{}
% puts 3 of 4 authors names on the first line, and the last on the second
% line, try using \AND instead of \And before the third author name.

\newcommand{\fix}{\marginpar{FIX}}
\newcommand{\new}{\marginpar{NEW}}

\nipsfinalcopy % Uncomment for camera-ready version

\begin{document}


\maketitle

\begin{abstract}
Analysis problems
\end{abstract}

\section{Solutions to the problems}

\begin{question}[]
\end{question}
\begin{solution}
We wish to find the minimum value of the given expression:
\eQb
\dfrac{(x + \dfrac{1}{x})^6 - (x^6 + \dfrac{1}{x^6}) - 2}{(x+\dfrac{1}{x})^3 + 
(x^3 + \dfrac{1}{x^3})},
\eQe
where $x > 0$. Observe that the following identity which follows from the
difference of squares:
\eQb
((x+\dfrac{1}{x})^3 + (x^3 + \dfrac{1}{x^3}))((x+\dfrac{1}{x})^3 - (x^3 +
\dfrac{1}{x^3})) &=& (x+\dfrac{1}{x})^6 - (x^3 + \dfrac{1}{x^3})^2 \\
&=& (x+\dfrac{1}{x})^6 - (x^6 + \dfrac{1}{x^6}) -2.
\eQe
Dividing the both side of the identity by $(x+\dfrac{1}{x})^3 + (x^3 + 
\dfrac{1}{x^3})$, we see that the given expression is equivalent to:
\eQb
(x+\dfrac{1}{x})^3 - (x^3 + \dfrac{1}{x^3}), 
\eQe
which can be further simplified to $3(x + \dfrac{1}{x})$. As $x>0$, by
the AM-GM inequality we see that the minimum happens at $x=1$, thereby showing that
the minimum of the given expression with $x>0$ is $3$. $\qed$ 
\end{solution}

\begin{question}
\end{question}
\begin{solution}
We wish to show that 
$(\prod_{i=1}^n a_i)^{\frac{1}{n}}+(\prod_{i=1}^n b_i)^
{\frac{1}{n}} \leq (\prod_{i=1}^n (a_i + b_i))^{\frac{1}{n}}$
Observe that the AM-GM inequality yields
\eQb
(\prod_{i=1}^{n} \dfrac{a_i}{a_i + b_i})^{\frac{1}{n}}
&\leq& \dfrac{1}{n} \prod_{i=1}^{n} \dfrac{a_i}{a_i + b_i}, \\ 
(\prod_{i=1}^{n} \dfrac{b_i}{a_i + b_i})^{\frac{1}{n}}
&\leq& \dfrac{1}{n} \prod_{i=1}^{n} \dfrac{b_i}{a_i + b_i}.
\eQe 
Adding the two inequalities, we obtain
\eQb
(\prod_{i=1}^{n} \dfrac{a_i}{a_i + b_i})^{\frac{1}{n}} + 
(\prod_{i=1}^{n} \dfrac{b_i}{a_i + b_i})^{\frac{1}{n}}
&\leq& \dfrac{1}{n} \prod_{i=1}^{n} \dfrac{a_i + b_i}{a_i + b_i}.
\eQe
We observe that the RHS of the inequality simplifies to $1$. Multiplying
the both sides by $(a_i + b_i)^{\frac{1}{n}}$, we get 
\eQb
(\prod_{i=1}^n a_i)^{\frac{1}{n}}+(\prod_{i=1}^n b_i)^
{\frac{1}{n}} \leq (\prod_{i=1}^n (a_i + b_i))^{\frac{1}{n}},
\eQe 
as desired. $\qed$
\end{solution}

\begin{question}
\end{question}
\begin{solution}
We wish to integrate $\int_{2}^{4} \dfrac{\sqrt{ln(9-x)}dx}{\sqrt{ln(9-x)} 
+ \sqrt{ln(x+3)}}$. We can rewrite the given integral as 
\eQb
\int_{-1}^{1} \dfrac{\sqrt{ln(6-x)}dx}{\sqrt{ln(6-x)} + \sqrt{ln(x+6)}}. 
\eQe
We can now separate the integral to two parts as
\eQb
\int_{0}^{1} \dfrac{\sqrt{ln(6-x)}dx}{\sqrt{ln(6-x)} + \sqrt{ln(x+6)}}
+  
\int_{-1}^{0} \dfrac{\sqrt{ln(6-x)}dx}{\sqrt{ln(6-x)} + \sqrt{ln(x+6)}}.
\eQe
We now use the substitution $u = -x$, so that $du = -dx$ on the second integral.
The integral becomes
\eQb
\int_{0}^{1} \dfrac{\sqrt{ln(6-x)}dx}{\sqrt{ln(6-x)} + \sqrt{ln(x+6)}}
+  
\int_{0}^{1} \dfrac{\sqrt{ln(x+6)}dx}{\sqrt{ln(6+x)} + \sqrt{ln(x-6)}}.
\eQe
which can be simplifed to $\int_{0}^{1} 1dx = 1$. $\qed$
\end{solution}

\bigskip

\begin{question}
\end{question}
\begin{solution}
We want to show the following inequality: 
\eQb
\dfrac{(m+n)!}{(m+n)^{m+n}} < \dfrac{m!}{m^m}\dfrac{n!}{n^n}.
\eQe
such that $m$ and $n$ are positive integers. Multiplying both sides 
by $\dfrac{(m+n)^{m+n}m^m n^n}{m!n!}$, we see that
the given inequality is equivalent to
\eQb
\dfrac{m^m n^n (m+n)!}{m!n!} < (m+n)^{m+n}.
\eQe
We can rewrite the inequality in terms of combinations as 
\eQb
{m+n \choose m}m^m n^n < (m+n)^{m+n}.
\eQe 
We can see from the binomial identity that the RHS contains the LHS term as one 
of its terms in the binomial expansion. As $m$ and $n$ are restricted
to be positive integers, the RHS will always contain another term with the
${m+n \choose 0}$ coffecient, hence acheiving a strict inequality. Hence,
we have shown that the given equality holds.
\end{solution}

\begin{question}[1999-A3]
\end{question}
\begin{solution}
\end{solution}




\end{document}
